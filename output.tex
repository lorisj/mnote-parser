\documentclass[12pt]{article}
\usepackage{notespkg}
\usepackage{screenread} %Puts every document on one page comment out if not needed.
%\addbibresource{bibliography.bib} % Rename bibliography.bib to your current .bib file, in the same directory. 
\usepackage{categorytheory}
\title{\currfilebase} % \currfilebase removes the .tex
\author{Loris Jautakas}
\begin{document}
\maketitle

%\tableofcontents


\section{Introduction}

This note covers basic definitions and results of category theory. It mostly follows \cite{riehl_2016}, but also contains notes from \cite{simmons_2011},
as well as special topics from \cite{nourani_2014}.


\begin{definition}{Category}{Category}
	A category $C$ consists of
	\begin{itemize}
		\item
		 A class $\Ob(C)$ consisting of objects
		\item
		 A class $\Hom(C)$ of morphisms.
		
		\begin{definition}{Morphism}{Morphism}
			A morphism is any object that has a source object $A \in \Ob(C)$ and a target $B \in \Ob(C)$. Morphisms are sometimes called arrows. 
			
			If $f$ is a morphism with source $A \in \Ob(C)$ and target $B \in \Ob(C)$, then this is usually written as $f:A \to B$. 
		\end{definition}
		\item
		
		A binary operation $\circ : M \to M$, called composition, which satisfies:
		\begin{enumerate}
			\item
			 
			$\circ$ is associative
			\item
			 $\Hom(A)$ has an idenity morphism for every $A \in \Ob(C)$
			\begin{definition}{Identity Morphism}{Identity Morphism}
				For every object $A \in \Ob(C)$, there exists an identity morphism $1_A : A \to A$, such that for every morphism $f:A \to B$:
				\begin{niceeq}
					f \circ 1_A = f = 1_B \circ f  
				\end{niceeq}
			\end{definition}
		\end{enumerate}
	\end{itemize}
\end{definition}
\begin{example}{Common Categories}{Common Categories}
	\begin{itemize}
		\item
		 
		$\cat{Set}$ has objects consisting of all sets, and morphisms consisting of all functions between sets.
		\item
		 
		$\cat{Top}$ has objects consisting of all topological spaces, and morphisms consisting of all continuous functions between these spaces.
		\item
		 
		$\cat{Group}$ has objects consistig of all groups, and morphisms consisting of all homomorphisms between groups.
		\item
		 
		$\cat{Mod}_R$ for a fixed ring $R$ (with identity),  is the category of left $R$-modules and $R$-module homomorphisms.
		If $R$ is a field, then we call this   
		\item
		 
		$\cat{Graph}$ has objects consisting of all graphs, and morphisms consisting of graph homomorphisms.
		\item
		 
		$\cat{Model}_T$ for any language $\mathcal L$ and first order $\mathcal L$-theory $T$ is a category with objects as 
		$[\mathcal L, T]$-structures (i.e. $\mathcal L$-structures $\mathcal M$ that model $T$, so $\mathcal M \models T$).
	\end{itemize}
\end{example}

\begin{result}{Unique Identity}{Unique Identity}
	Identity morphisms in a category are unique.
	\dline
	\begin{proof}
		Consider an object $A$ with two identity morphisms $f, g : A \to A$. Then note $f = f \circ g = g$. 
		Thus $f = g$ and identity morphisms are unique.
	\end{proof}
\end{result}
\begin{definition}{Hom Class}{Hom Class}
	Let $C$ be a category. Let $A,B \in ob(C)$ be two objects. Denote $C(A,B) = \{ f \in \Hom(C)  |  f : A \to B\}$, 
	i.e. the class containing all morphisms with source $A$ and target $B$. This is called the Hom-class, and is sometimes written as $\Hom(A,B)$.
\end{definition}

\begin{definition}{Isomorphism}{Isomorphism}
	A morphism $f:X\to Y$ is an isomorphism if and only if it is invertible, i.e there exists some $g:Y \to X$ such that:
	\begin{niceeq}
		g\circ f &= 1_X\\
		f\circ g &= 1_Y
	\end{niceeq}
	
	We then say two objects $X,Y$ are isomorphic.
	
\end{definition}

\begin{definition}{Endomorphism}{Endomorphism}
	An endomorphism is a morphism whose domain is the same as the codomain, i.e. $f:X \to X$ is an endomorphism.
	a set of all endomorphisms of an object $X$ is denoted $\End(X)$.
\end{definition}
\begin{definition}{Automorphism}{Automorphism}
	A automorphism is a morphism which is both an isomorphism and an endomorphism.
\end{definition}


\begin{example}{Category Isomorphisms}{Category Isomorphisms}
	Note that morphisms are technically binary relations, (if they arent a set then they can be though of as a relation of a class) 
	but this sometimes is not the right way of looking at them.
	This is true in the following example: 
	\begin{enumerate}
		\item
		 For any ring $R$, define the category $C$:
		\begin{itemize}
			\item
			 $\Ob(C) \eqdef \Z_+$
			\item
			 $\Hom(C)\eqdef$ the set of $C(n,m)= R^{n \times m}$, i.e. all $n$ by $m$ matrices.
			\item
			 $\circ \eqdef$ matrix multiplication
		\end{itemize}
		To check this forms a category, note that:
		\begin{itemize}
			\item
			 $\circ$ is associative because matrix multiplication is associative
			\item
			
			Every object has an identity, namely for any $n \in \Ob(C)$ there is the $n\times n$ identity matrix $I_n$, which has the property that
			for any morphism $f:m \to n$ (i.e. for every $n \times m$ matrix) we have $I_n \circ f = f \circ I_m = f$.
		\end{itemize}
		
		
		Thus $C$ is a category. 
		
		Note that while technically $\Hom(C)$ consists of relations, (i.e. you have a relation for each $n\times m$ matrix)
		it is not productive to think of morphisms this way, so you should rather think of morphisms as some new object, i.e. an arrow.
		\item
		
		For any monoid $\mathcal M = (M,*)$, define the category $C=\cat{B}_M$:
		\begin{itemize}
			\item
			 $\Ob(C)$ consists of some single object (could be anything, let's call it $o$)
			\item
			 For every monoid element $m \in M$, define a morphism $f_{m}: o \to o$.
			\item
			 Define $\circ$ as the binary operation $f_m \circ f_n \mapsto f_{m * n}$.
		\end{itemize}
		
		Note that monoids have identity elements and associative binary operation.
	\end{enumerate}
\end{example}

\end{document}
